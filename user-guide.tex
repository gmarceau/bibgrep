\documentclass[11pt]{article}
\usepackage{epsfig}
\usepackage{fullpage}
\title{Bibgrep - an indexing and searching \\ tool for BibTex files}
\author{Guillaume Marceau (gmarceau@cs.brown.edu)}
\setlength{\parindent}{0pt}
\setlength{\parskip}{7pt}
\linespread{1.2}
\begin{document}
\maketitle

\section*{User manual}

   Bibgrep searches the named input BibTex files for entries matching
   a given query. Its usage is similar to the command ``grep''.


   Bibgrep will create an index for each BibTex file it touches, and
   save the result in {\tt $\sim$/.bibgrep.idx } (by defaults) to
   speedup future queries to the same files. Bibgrep watches the
   modification date and the size of the original BibTex file and will
   update (and delete) its index whenever needed.  Bibgrep searches
   within the BibTex files mentioned on the command line and only
   theses files, even when using an index files.


   The query language is an extension of the best-known query 
   language. It uses the {\em key colon word} convention used at google.com
   (everybody's favorite search engine).


   Enter a number of keywords and bibgrep will return entries that
   include all the search terms (automatic ``and'' queries). For example,
   to search for McGill's effort in robotic soccer, query:

\begin{verbatim}
        robocup mcgill
\end{verbatim}

   On the command line, you would type : ``{\tt bibgrep 'robocup mcgill'
   file.bib}''

   You can also exclude a word by adding a minus sign (``-'') in front
   of the term to avoid. For example, to search for green trees,
   rather than algorithmic ones, query:


\begin{verbatim}
        tree -binary -search
\end{verbatim}


   To restrict a word to a specific field, write the field name, a colon,
   then the search term (all without spaces). For example, to search for
   George Bush's contributions to Science Magazine, query:


\begin{verbatim}
        author:George author:Bush title:Science
\end{verbatim}


   If the field name is omitted, it defaults to the previous field name
   explicitly mentioned. The example above can be rewritten as:


\begin{verbatim}
        author:George :Bush title:Science
\end{verbatim}


   It is also possible to achieve the same effect using Quotes
   (although this will change if searching per sentence is ever
   implemented) :


\begin{verbatim}
        author:''George Bush'' title:Science
\end{verbatim}


   The ``author'' field can also be searched by first or last name. Search
   using field ``firstname:'' and ``lastname:'' respectively. 


   Disjunctions (``or'' queries) are available as well. Alternatives
   should be separated with a double forward slash (``{\tt //}''), and
   either ``{\tt ()}'' ``{\tt \{\}}'' or ``{\tt []}'' can be used for
   grouping, as long as they match pairwise. Negation of disjunctions
   also works as expected. For example, to find delicious meal accompagnements,
   avoiding the usual suspects from France, query:


\begin{verbatim}
        (wine // champagne)  -(france // bordeaux)
\end{verbatim}


   It is possible use a exclamation mark (``{\tt !}'') instead of a dash. This is a
   useful for queries which begin with a negation, otherwise they would be
   mistaken for a unknown option. Note that some shells (bash) will insist on
   having a space between the ``{\tt !}'' and the following word.


\begin{verbatim}
        ! author:Knuth   ! author:Dykstra
\end{verbatim}


   Shell-style regular expressions are supported, both as search word and as
   field name. ``{\tt *}'' matches a sequence of characters, and ``{\tt ?}'' matches
   a single character or nothing. They are most useful to abbreviate field names
   and to search for the singular and plural of a word at once. To match titles
   with either ``computer'' or ``computers'' (and some other unlikely things),
   query:


\begin{verbatim}
        t*:computer?
\end{verbatim}


   Search terms are always case insensitive.


\section*{Command line arguments}
\linespread{1}
\small
\begin{verbatim}
--morehelp      shows this document
--all           work with all the already indexed BibTex files 
--index         specify an index file to use instead of the defaults (~/.bibgrep.idx) 
--interactive   present a prompt for interactive querying 
--summaries     print one line summaries instead of the full BibTex entries, 
                in a format compatible with emacs' ``M-x compile'' command 
--sort          set the sorting field 
--list          list the files indexed in the index file 
--forget        remove the files given on the command line from the index 
--no-save       do not save the modifications done to the index 
--no-load       create an index anew, potentially clobbering the existing file) 
--no-update     do not update the index, even if some BibTex files have changed 
--verbose       comment the progress of the search (on stderr)
--              use the rest of the command line as a query (rather
                than at its usual place) 
 
--no-index      short for `--no-load --no-save' 
-i              short for --interactive 
-sm             short for --summaries 
-st             short for --sort 
-l              short for --list 
-r              short for --forget 
-v              short for --verbose 
 
-help           display the list of options 
--help          display the list of options

\end{verbatim}
\normalsize
\linespread{1.2}

\section*{Using bibgrep interactively}

The {\tt --interactive} option makes bibgrep present a prompt for
interactive querying. This allow to make multiple search through 
a large index without having to reload it from disk each time. The
commands available in interactive mode are :

\small
\linespread{1}
\begin{verbatim}
find <query>       searches for BibTex entries.
add <file> ...     add BibTex file(s) to list of files to search
remove <file> ...  remove BibTex file(s) from files to search
targets            list the file targeted for querying

index <file> ...   add file(s) to the index
forget <file> ...  remove file(s) from the index
list               list the indexed files
savenow            save the index to disk now

help               the this list of commands
morehelp           shows this document

exit               leave bibgrep
quit               leave bibgrep
\end{verbatim}
\linespread{1.2}


The following commands changes options :

\linespread{1}
\begin{verbatim}
  sort [ off | <field name> ]   sets the sort key
  summaries [ on | off ]        toggle one-line summaries
  update [ on | off ]           turns off automatic synchronization of the index
  save [ on | off ]             disable saving the index when quitting
  verbose [ on | off ]          become chatty about the search process (on stderr)
\end{verbatim}
\linespread{1.2}
Typing the name of an option without the argument shows its
current status

\section*{Copyright}

\scriptsize
{\tt
Copyright (C) 2003 Guillaume Marceau <gmarceau at cs dot brown dot edu>

This program is free software; you can redistribute it and/or
modify it under the terms of the GNU General Public License
as published by the Free Software Foundation; either version 2
of the License, or (at your option) any later version.

This program is distributed in the hope that it will be useful,
but WITHOUT ANY WARRANTY; without even the implied warranty of
MERCHANTABILITY or FITNESS FOR A PARTICULAR PURPOSE.  See the
GNU General Public License for more details.

You should have received a copy of the GNU General Public License
along with this program; if not, write to the Free Software
Foundation, Inc., 59 Temple Place - Suite 330, Boston, MA  02111-1307, USA.
}
\normalsize

\end{document}


